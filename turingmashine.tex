\section{Turing Maschine}
\label{turingmaschine}
Die Turingmaschine wurde 1936 von Alan Turing entwickelt und ist rein theoretisch, denn eine Maschine exakt wie es Turing beschreibt wurde niemals gebaut, da dies auch nicht sinnvoll wäre. Turing entwickelte diese Theorie aufgrund des Entscheidungsproblem von David Hilber und Wilhelm Ackermann \cite{theessentialturing}. Die Turing Maschine ist die heutige Grundlage vieler Programmiersprachen, wie Java, C++ oder Python. Diese Programmiersprachen werden auch als \textit{"Turing complete"} bezeichnet, was so viel heißt wie, dass die Turing Maschine diese theoretischerweise emulieren kann.

\subsection{Deterministische Turingmaschine}
Die Theorie der deterministischen Turingmaschine beschreibt die Aktionen der Turingmaschine als etwas, dass von Anfang an nur einen einzigen klaren Weg hat, sodass man theoretischerweise vor Programmstart sagen kann, welchen Weg die Turingmaschine nimmt. Diese Theorie ist auf alle Programme anwendbar, die Turing complete sind.

\subsection{Nichtdeterministische Turingmaschine}
Die Theorie der nichtdeterministischen Turingmaschine beschreibt keinen klaren von Anfang an festgelegten Weg, sondern vielmehr ein Programm, dass viele verschiedene mögliche Wege hat um ans Ziel zu kommen. Dieses Modell ist allerdings rein theoretisch und nach dem heutigen Wissensstand nicht realisierbar.

\subsection{Funktion}
Die Turing Maschine beinhaltet lediglich zwei Bauteile.
\begin{enumerate}
	\item  Ein unendlich langes Band, dass in eine Kette von horizontalen Kästchen eingeteilt ist. Jedes dieser Kästchen kann entweder eine 0, eine 1 oder auch nichts beinhalten. Zudem kann das Band nach rechts und nach links verschoben werden.
	\item Ein Kasten, auch "Heap" oder auch "Scanner" genannt, welcher die Zahlen auf dem Band einlesen, löschen und ändern kann.
\end{enumerate}

\begin{figure}[hbtp]
	\centering
	\includegraphics[scale=1]{TuringmashinePicture.png}
	\caption{Beispielhafte Darstellung einer Turing Maschine\cite{theessentialturing}}
\end{figure}

\subsection{Beispiel}
Zur besseren Vorstellung, wird das Ganze an einem Beispiel erklärt. Nehmen wir mal an, wir wollen, dass die Turing Maschine für uns unendlich hoch zählt. So kann man dies mit nur zwei einfachen Befehlen ausführen.

\begin{itemize}
	\item Befehl 1: Bei einer 1, ändere diese in eine 0 und gehe nach links
	\item Befehl 2: Bei einer 0 oder einer Lücke, ändere diese in eine 1 und gehe zum letzten Bit der Zahl
\end{itemize}

Die Befehle für die Turing Maschine kann man sich also heute wie die Programme auf unseren Computern vorstellen.

\subsection{Codebeispiel}
Dieser Code aus Java macht praktisch genau dasselbe, wie das kleine Beispielprogramm der Turing Maschine. Das Einzige was auffällt ist, dass dieses Programm nur bis 1000 zählt, da unsere Turing Maschine einen unendlichen Speicher besitzt, anders als jeder PC auf der Welt.

\begin{minted}{java}
for (int i = 0; i < 1000; i++) {}
\end{minted}

In der for-Schleife wird zuerst die Variable i mit dem Datentyp Integer versehen und auf null gesetzt. Hinter dem ersten Semikolon steht dann die Bedingung, solange die Variable i kleiner wie 1000 ist. Hinter dem zweiten Semikolon steht dann die Anweisung dafür, was passiert solange die Bedingung erfüllt ist, das i++ bedeutet, dass die Variable i um eins hochgezählt wird. 

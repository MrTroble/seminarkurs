\section{Lambda calculus}
Der Lambda calculus ist ein Datenverarbeitungskonzept erfunden von Alonzo Church, dem Professor von Alan Turing. Turing studierte unter Church einige Zeit an der Princeton Universität bevor er dann wieder zurück nach England ging.
\subsection{Alonzo Church}
\subsection{Gesetze des Lambda calculus}
Der Lambda calculus besteht aus 3 Grundgesetzen.
\begin{itemize}
\item Funktionen können erstellt werden.
\item diese können/müssen Daten Ein und Ausgeben.
\item Man weiß nicht was die Funktionen tun.
\end{itemize}
Ausführlicher: Es muss ein Weg vorhanden sein eine Funktion zu definieren, dafür gibt es eine Mathematische Definition. Die Erzeugung einer Funktion ist aber nicht auf die Mathematische Notation beschränkt, ganz im Gegenteil, die Notationen im Modernen Lambda Programming weichen, nicht zuletzt aus technischen Gründen, stark von der Mathematischen Notation ab. Dabei MÜSSEN im originalen Lambda Ein- und Ausgaben definiert werden. Dies weicht im modernen Lambda ebenfalls von der originalen Definition ab. Hier können Ein- und Ausgaben definiert werden, müssen es aber nicht. Dies führt zu mehr Flexibilität, so können die Lambda Funktionen auch an stellen Eingesetzt werden wo nicht direkt Daten verarbeitet werden. Als letztes kennt man beim Ausführen der Funktion den Inhalt der Funktion nicht. Das ist der große unterschied zu anderen Mathematischen Funktionen. Wie genau dies Funktioniert wird im übernächsten Kapitel besser klar.
\subsection{Mathematische Notation}
\begin{eqnarray}
•
\end{eqnarray}
\subsection{Lambda in Funktions basierten Programmiersprachen }
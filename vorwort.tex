\section{Vorwort}
Der Vater des Computer; die Person, die den zweiten Weltkrieg wahrscheinlich um zwei Jahre verkürzte. Was gibt es da noch zu sagen? Alan Turing ist einer der bekanntesten und einflussreichsten Mathematiker des letzten Jahrhunderts. Er beeinflusst nicht nur die Mathematik und heutige Informationstechnik durch die Turingmaschine, sondern auch die Philosophie, durch zum Beispiel den Turing Test. In seiner meist zitierten Arbeit ging es sogar um Biologie. Am bekanntesten, nicht zuletzt durch den Film 'The Imitation Hame', ist jedoch seine Beteiligung am zweiten Weltkrieg, durch die Entschlüsselung der Enigma der Nazis. Seine Arbeit im Bereich Computertechnik, sowie in Philosophie, sein Papier über Maschinen und Intelligenz, sind mehr als nur interessant. Turing hat mit diesen Werken unsere heutiges Leben maßgeblich beeinflusst. In Kapitel \ref{historie} wird die Biografie rund um Alan Turing beleuchtet. Hierzu wird auch etwas auf den geschichtlichen Kontext eingegangen. Des Weiteren werden ein paar seiner Werke hier zeitlich eingeordnet. Darauf in Kapitel \ref{enigma} erfährt man alles rund um die Verschlüsselungsmaschine Enigma, wie sie funktioniert, aus welchen Teilen diese besteht und warum sie geknackt wurde. Des Weiteren wird auf die Geschichte der Enigma, so wie Turings Arbeiten an der Entschlüsselung mit Hilfe der Bombe eingegangen. In Kapitel \ref{turingmaschine} wird auf das theoretische Datenverarbeitungssystem von Alan Turing eingegangen. Abschließend in Kapitel \ref{turingtest} geht es um die Frage, wie man Menschen von Maschinen unterscheidet und ob Maschinen denken können. Wie wichtig diese Fragen geworden sind und noch weiter werden, sowie die Einflüsse Turings Arbeiten in der Philosophie, werden hier auch beleuchtet.
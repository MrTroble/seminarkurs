\section{Vorwort}
Der Vater des Computer, die Person die wahrscheinlich den zweiten Weltkrieg um zwei Jahre verkürzte. Was gibt es da noch zu sagen? Alan Turing eine der bekanntesten und einflussreichsten Mathematiker des letzten Jahrhunderts. Er beeinflusst nicht nur die Mathematik und heutige Informationstechnik, durch die Turingmaschine, sondern auch die Philosophie, durch zum Beispiel den Turing Test. In seinem meist zitierten Papier ging es sogar um Biologie. Am bekanntesten, nicht zuletzt durch den Film 'the imitation game', ist jedoch seine Beteiligung am zweiten Weltkrieg, durch die Entschlüsselung der Enigma der Nazis. Seine Arbeit im Bereich Computertechnik sowie in Philosophie, sein Papier über Maschinen und Intelligenz, sind mehr als nur interessant. Turing hat mit diesen Werken unsere heutiges Leben maßgeblich beeinflusst. In den folgenden Kapiteln wird die tragische Geschichte rund um Alan Turing und ein paar seiner Zeitgenossen beleuchtet. Die Turing Maschine, die Turing Bomb, der Turing Test und seine Auswirkungen auf die Zukunft, sowie die Turing-Church-Theses werden in den Folgenden Abschnitten beleuchtet.
\section{Historie}
Alan Mathison Turing wurde 1912, am 23 Juni in London geboren.Er starb am 7 Juni 1954 im Alter von 42 Jahren an seinem Wohnort in Wilmslow in Cheshire. Er machte seinen Abschluss an der Sherborne School in Dorset und besuchte anschließend ab Oktober 1931 das King's College in Cambridge, an dem er Mathematik studierte. Er beendete sein Studium 1934 und im März 1935 arbeitete er im Alter von 22 Jahren bereits als Dozent am King's College. 1936 veröffentlichte er eine seiner wichtigsten Werke "On Computable Numbers, with an Application to the Entscheidungsproblem", welche das Prinzip der Turing Maschine beinhaltet. Durch die Unterstützung von John von Neumann und Max Newman arbeiteten bereits mehrere Gruppen 1945 an der Umsetzung der Turing Maschine in Hardware, also an der Umsetzung des ersten Computers. Turing verließ 1936 Cambridge um seine Forschungen an der Princeton University weiter zu führen und veröffentlichte 1938 "Systems of Logic Based on Ordinals". 
Im Sommer 1938 kehrte Turing als Dozent an die King's zurück bis er beim Ausbruch des Krieges im September 1938 in den Bletchley Park zog, ein millitärisches Lager zur Entschlüsselung deutscher Nachrichten. Turings exzellente Arbeit hatte Kriegsentscheidente Konsequenzen. Turing war Hauptentwickler der Turing Bombe, eine extrem schnelle Entschlüsselungsmaschine. Im Nachhinein wird vermutet das die Arbeiten von Turing und seinen Kollegen den Krieg in Europa um mindestens 2 Jahre verkürzt hat.
Als der Krieg 1945 beendet war, ging Turing nach London ins National Physical Laboratory um seine Theorie der Turing Maschine in Hardware umzusetzen. Im Februar 1947 veröffentlichte er "Lecture on the Automatic Computing Engine" in der er als Erster etwas über Computerintelligenz schreibt und in seinem technischer Bericht 1948 "Intelligence Machinery" definiert er den Begriff 'Artificial Intelligence' kurz auch 'AI'. Im zwei Jahre später erschienenen Artikel "Computing Machinery and Intelligence" behandelt er den heute bekannten Turing Test, der ein Kriterium ist ob Maschinen denken können oder nicht.
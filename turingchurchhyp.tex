\section{Church-Turing Thesis}
In diesem Kapitel geht es um die Church-Turing Thesis, was sie besagt und was für Auswirkungen sie hat. Außerdem wird im Zuge dessen die Turing Maschine und der Lambda calculus gegenübergestellt.
\subsection{Die These}
Die These besagt, dass alle möglichen Rechnungen bzw. Datenverarbeitungen mit dem Modell der Turing Maschine dargestellt werden können. Des weiteren ist der Lambda calculus mit der Turing Maschine gleich zu stellen, das heißt, dass ebenfalls jeder Datensatz mit Hilfe diese Paradigma verarbeitet werden können. Beide Systeme sind also nur Unterschiedliche Darstellungsweisen für Rechnungen bzw. Datenverarbeitung.
\cite{sep-church-turing}
\subsection{Gegenüberstellung}
Wie bereits erwähnt hat der Lambda calculus keinen internen "State", dies steht im Kontrast zur Turing Maschine diese hat einen internen "Speicher". Damit  
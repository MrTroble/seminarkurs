\section{Turingtest}
In diesem Abschnitt geht es um Turings Ausflug in die Philosophie und sein Werk "Computing Machinery and Intelligence". Dabei wird die Frage betrachtet ob Maschinen denken können und wie man eine Maschine von einen Menschen unterscheiden kann, sowie die Probleme die dabei entstehen und heutige Anwendungen.
\subsection{The Imitation Game}
Turing stellt eine bessere Frage in den Raum als die oben Angegeben. Um die Frage allerdings Stellen zu können muss ich das "Imitation Game" einführen. Das Spiel funktioniert wie Folgt. Lass A ein Mann sein, B eine Frau und C einen Detektiv. Der Detektiv (C) will also heraus finden wer welches Geschlecht hat. Ohne dabei die Sprache zu hören oder die Person zu sehen. Um dafür zu sorgen das dies Passiert gehen wir davon aus, dass der Detektiv in einem anderen Raum ist wie A und B, die einzige Kommunikation die besteht ist also Textbasierte. Also Versucht der C Fragen zu verwenden um mit Hilfe der Antworten der beiden Personen zu bestimmen welche weiblich und welche Männlich ist. Dies wendet Turing nun auf Maschinen an, dabei beschränkt er sich auf digital Computer. Es wird versucht auf äußerliche Fragen zu verzichten, da dies zu keinem eindeutigen Ergebnis führen würde. Die Maschine könnte ganz einfach ein eigenes Aussehen annehmen, und die Person kennt ihres, daher ist es schwierig anhand der Antworten eindeutig zu Unterscheiden. Deshalb wird versucht auf Logische und Fragen einzugehen. Beispiel: Subtrahiere 56 von 18934.\cite{computing} Diese Spiel wird nun nicht nur dazu genutzt eine Maschine von einem Menschen zu Unterscheiden sondern, es wird impliziert die Frage gestellt ob eine Maschine denken kann. Wenn man ihre Antworten nämlich nicht mehr voneinander unterscheiden kann muss man sich Gedanken darüber machen ob wir dann nicht Intelligentes Leben geschaffen haben könnten und wie wir diese dann Behandeln.
\subsection{Chinese room problem}
Eine der berühmtesten Problem des Turing Tests ist das chinese room problem. Dieses Problem stellt in Frage ob man mit dem Turingtest wircklich die Intelligenz eines Lebewesens feststellen kann. Dabei wird eine weitere Situation angenommen. Nehmen wir also an zwei Personen sitzen sich gegen über in einem Raum. Die eine Person ist eine Chinese die andere Person ein Deutscher. Auf dem Tisch liegt nun ein Deutsch-Chinesisch Wörterbuch. Die beiden Personen versuchen Miteinander zu Kommunizieren. Dazu nutzt der Deutsche ein Wörterbuch. Nun stellt sich die Frage, wenn der Deutsche nun seinen Satz auf Chineesisch übersetzt, ob er wirklich weiß was er sagt. Übertragen auf eine Maschine stellt sich nun ebenfalls die Frage, kann die Maschine wirklich wissen was sie tut und diese Aktionen auch selber hinterfragen. Darüber hinaus stellt sich die Frage wie Intelligent Maschinen wirklich werden können.
\subsection{Anwendungen in der Moderne}
Hier folgt eine kurze Beleuchtung des heutigen Einsatzfeldes des Turing Tests. Dieser wird in der Internetsicherheit sehr oft verwendet um automatisierte Anfragen zu filtern. Als Beispiel wird hier exemplarisch ReCaptcha von Google Beleuchtet. Es gibt aber genug andere sogenannte Captcha Methoden. Diese geben eine Art von Fragen an hierbei ist die Kommunikation allerdings nicht nur, wie im Turing, auf Textbasierte Kommunikation beschränkt. Im Gegenteil hier werden sogar bewusst Bilder eingesetzt, damit der zu testende Computer bzw. die Person die den Computer bedient, eine eine Frage zu bzw mit diesen Bildern beantworten muss. Dabei sind die Antworten auf die Eigentliche Frage nicht einmal so wichtig wie es einem im ersten Moment erscheint. Die Ironie, und eine weiter Abweichung, ist dabei das Mann hier gegen ein Neuralesnetzwerk "spielt". Dieses muss anhand der Vorliegenden Informationen bestimmen ob die Anfrage von einem Mensch kam oder ob es sich dabei nur um eine automatisch generierte Anfrage handelt. Wie genau dieses Netzwerk genau entscheidet weiß wahrscheinlich nicht mal Google. Wichtig ist aber welche Daten dem Netzwerk dabei zur Verfügung gestellt werden um seine Entscheidung zu treffen. Für weitere Infos verweise ich hier auf einen Blog-Eintrag\cite{captcha}. Zu den Auszuwertenden Daten gehören unter anderem die Mausbewegung sowie die gesendeten Daten des Browsers (z.B. User-Agent, IP, Cookies).
\subsection{Auswirkungen auf Kunst und Gesellschaft}
Diese Grundsätzlichen Gedanken und die Frage ob Maschinen überhaupt denken können hat weitreichende Auswirkung auf die Moderne Kunst, darunter vor allem Computerspiele. Sehr gute Beispiele sind das gleichnamigen Spiel "The Turing Test" oder auch "The Talos Principle" in denen der Turing Test sehr häufig vorkommt. Es ist wichtig hier darauf aufmerksam zu machen, dass das Imitation Game heute noch für sehr viel Kopfzerbrechen unter den Wissenschaftlern, Philosophen so wie in der gesamten Gesellschaft auslöst. Wir werden bald an den Zeitpunkt kommen wo es wichtig sein wird, sich mit dieser Frage auseinander zu setzen. Weiterführend dazu siehe The future of the mind: Exploring machine consciousness von Dr. Susan Schneider\cite{explorecons}


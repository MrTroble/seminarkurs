\section{Turingtest}
\label{turingtest}
In diesem Abschnitt geht es um Turings Ausflug in die Philosophie und sein Werk "Computing Machinery and Intelligence". Dabei wird die Frage betrachtet ob Maschinen denken können und wie man eine Maschine von einen Menschen unterscheiden kann. Sowie heutige Anwendungen und Auswirkungen auf Kunst und Gesellschaft.
\subsection{The Imitation Game}
Turing stellt eine bessere Frage in den Raum, als die Frage "Können Maschinen Denken?". Um die Frage allerdings Stellen zu können muss ich das "Imitation Game" einführen. Das Spiel funktioniert wie folgt. Lass A ein Mann sein, B eine Frau und C einen Detektiv. Der Detektiv (C) will also heraus finden wer welches Geschlecht hat, ohne dabei die Sprache zu hören oder die Person zu sehen. Um diesen Zustand zu gewährleisten gehen wir davon aus, dass der Detektiv in einem anderen Raum ist wie A und B, die einzige Kommunikation die besteht ist also textbasiert. Also versucht C Fragen zu verwenden, um mit Hilfe der Antworten beider Personen zu bestimmen, welche weiblich und welcher Männlich ist. Dies wendet Turing nun auf Maschinen an, dabei beschränkt er sich auf Digitalcomputer. Es wird versucht auf äußerliche Fragen zu verzichten, da dies zu keinem eindeutigen Ergebnis führen würde. Die Maschine könnte ganz einfach ein eigenes fingiertes Aussehen annehmen und die Person kennt ihres, daher ist es schwierig anhand der Antworten eindeutig zu unterscheiden. Deshalb versucht C logische Fragen zu verwenden um heraus zu finden ob es sich um eine reale Person handelt. Beispiel: Subtrahiere 56 von 18934.\cite{computing} Dieses Spiel wird nun nicht nur dazu genutzt eine Maschine von einem Menschen zu unterscheiden, sondern es wird implizit die Frage gestellt ob eine Maschine denken kann. Wenn man ihre Antworten nämlich nicht mehr voneinander unterscheiden kann, muss man sich Gedanken darüber mache,  ob wir dann nicht intelligentes Leben geschaffen haben könnten und wie wir dieses dann behandeln.
\subsection{Chinese room problem}
Eine der berühmtesten Probleme des Turing Tests ist das chinese room problem. Dieses Problem stellt in Frage, ob man mit dem Turingtest wirklich die Intelligenz eines Lebewesens feststellen kann. Dabei wird eine weitere Situation angenommen. Nehmen wir also an zwei Personen sitzen sich gegenüber in einem Raum. Die eine Person ist ein Chinese die andere Person ein Deutscher. Auf dem Tisch liegt nun ein Deutsch-Chinesisch Wörterbuch. Die beiden Personen versuchen miteinander zu kommunizieren. Dazu nutzt der Deutsche ein Wörterbuch. Nun stellt sich die Frage, wenn der Deutsche nun seinen Satz auf Chinesisch übersetzt, ob er wirklich weiß was er sagt. Übertragen auf eine Maschine stellt sich nun ebenfalls die Frage, kann die Maschine wirklich wissen was sie tut und diese Aktionen auch selber hinterfragen. Darüber hinaus stellt sich die Frage wie intelligent Maschinen wirklich werden können.
\subsection{Anwendungen in der Moderne}
Hier folgt eine kurze Beleuchtung des heutigen Einsatzfeldes des Turing Tests. Dieser wird in der Internetsicherheit sehr oft verwendet um automatisierte Anfragen zu filtern. Als Beispiel wird hier exemplarisch ReCaptcha von Google beleuchtet. Es gibt aber genug andere sogenannte Captcha Methoden. Diese geben eine Art von Frage an, hierbei ist die Kommunikation allerdings nicht nur wie im Turing Test, auf textbasierte Kommunikation beschränkt. Im Gegenteil, hier werden sogar bewusst Bilder eingesetzt, damit der zu testende Computer beziehungsweise die Person die den Computer bedient, eine Frage zu diesen Bildern beantworten muss. Dabei sind die Antworten auf die eigentliche Frage nicht einmal so wichtig wie es einem im ersten Moment erscheint. Die Ironie, und eine weitere Abweichung ist dabei, dass man hier gegen ein neuronales Netzwerk "spielt". Dieses muss anhand der vorliegenden Informationen bestimmen, ob die Anfrage von einem Mensch stammt oder ob es sich dabei nur um eine automatisch generierte Anfrage handelt. Wie genau dieses Netzwerk entscheidet weiß wahrscheinlich nicht mal Google. Wichtig ist aber welche Daten dem Netzwerk dabei zur Verfügung gestellt werden, um seine Entscheidung zu treffen. Für weitere Informationen verweise ich hier auf einen Blog-Eintrag\cite{captcha}. Zu den auszuwertenden Daten gehören unter anderem die Mausbewegungen sowie die gesendeten Daten des Browsers (z.B. User-Agent, IP, Cookies).
\subsection{Auswirkungen auf Kunst und Gesellschaft}
Diese grundsätzlichen Gedanken und die Frage, ob Maschinen überhaupt denken können hat weitreichende Auswirkung auf die Moderne Kunst, darunter vor allem Computerspiele. Sehr gute Beispiele sind, das gleichnamige Spiel "The Turing Test" oder auch "The Talos Principle" in denen der Turing Test sehr häufig vorkommt. Es ist wichtig hier darauf aufmerksam zu machen, dass das Imitation Game heute noch für sehr viel Kopfzerbrechen unter den Wissenschaftlern, Philosophen so
wie in der gesamten Gesellschaft auslöst. Wir werden bald an den Zeitpunkt kommen wo es wichtig sein wird, sich mit dieser Frage auseinander zu setzen. Weiterführend dazu siehe The future of the mind: Exploring machine consciousness von Dr. Susan Schneider\cite{explorecons}


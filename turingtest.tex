\section{Turingtest}
In diesem Abschnitt geht es um Turings Ausflug in die Philosophie und sein Werk "Computing Machinery and Intelligence". Dabei wird die Frage betrachtet ob Maschinen denken können und wie man eine Maschine von einen Menschen unterscheiden kann, sowie die Probleme die dabei entstehen und heutige Anwendungen.
\subsection{The Imitation Game}
Turing stellt eine bessere Frage in den Raum als die oben Angegeben. Um die Frage allerdings Stellen zu können muss ich das "Imitation Game" einführen. Das Spiel funktioniert wie Folgt. Lass A ein Mann sein, B eine Frau und C einen Detektiv. Der Detektiv (C) will also heraus finden wer welches Geschlecht hat. Ohne dabei die Sprache zu hören oder die Person zu sehen. Um dafür zu sorgen das dies Passiert gehen wir davon aus, dass der Detektiv in einem anderen Raum ist wie A und B, die einzige Kommunikation die besteht ist also Textbasierte. Es wird versucht auf äußerliche Fragen zu verzichten. Dies würde zu keine m eindeutigen Ergebnis führen, da die Maschine ganz einfach einen Aussehen annehmen kann, und die Person kennt ihres, d.h. ist es hier schwierig zu Unterscheiden. Deshalb wird versucht auf Logische und Ethische Fragen einzugehen. Beispiel: Subtrahiere 56 von 18934.
\subsection{Gegenpositionen}
\subsubsection*{}
\subsection{Chinese room problem}

\subsection{Anwendungen in der Moderne}
Hier folgt eine kurze Beleuchtung des heutigen Einsatzfeldes des Turing Tests. Dieser wird in der Internetsicherheit sehr oft verwendet um automatisierte Anfragen zu filtern. Als Beispiel wird hier exemplarisch ReCaptcha von Google Beleuchtet. Es gibt aber genug andere sogenannte Captcha Methoden. Diese geben eine Art von Fragen an hierbei ist die Kommunikation allerdings nicht nur, wie im Turing, auf Textbasierte Kommunikation beschränkt. Im Gegenteil hier werden sogar bewusst Bilder eingesetzt, damit der zu testende Computer bzw. die Person die den Computer bedient, eine eine Frage zu bzw mit diesen Bildern beantworten muss. Dabei sind die Antworten auf die Eigentliche Frage nicht einmal so wichtig wie es einem im ersten Moment erscheint. Die Ironie, und eine weiter Abweichung, ist dabei das Mann hier gegen ein Neuralesnetzwerk "spielt". Dieses muss anhand der Vorliegenden Informationen bestimmen ob die Anfrage von einem Mensch kam oder ob es sich dabei nur um eine automatisch generierte Anfrage handelt. Wie genau dieses Netzwerk genau entscheidet weiß wahrscheinlich nicht mal Google. Wichtig ist aber welche Daten dem Netzwerk dabei zur Verfügung gestellt werden um seine Entscheidung zu treffen. Für weitere Infos verweise ich hier auf einen Blog-Eintrag (https://www.security-insider.de/wie-funktioniert-ein-captcha-a-683209/). Zu den Auszuwertenden Daten gehören unter anderem die Mausbewegung sowie die gesendeten Daten des Browsers (z.B. User-Agent, IP, Cookies).
\subsection{Auswirkungen auf Kunst}
Diese Grundsätzlichen Gedanken und die Frage ob Maschinen überhaupt denken können hat weitreichende Auswirkung auf die Moderne Kunst, darunter vor allem Computerspiele. Sehr gute Beispiele sind das gleichnamigen Spiel "The Turing Test" oder auch "The Talos Principle" in denen der Turing Test sehr häufig vorkommt. Es ist wichtig hier darauf aufmerksam zu machen, dass das Imitation Game heute noch für sehr viel Gedanken 


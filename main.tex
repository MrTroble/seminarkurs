\documentclass[11pt,a4paper]{article}
\usepackage[utf8]{inputenc}
\usepackage{amsmath}
\usepackage{amsfonts}
\usepackage{amssymb}
\usepackage{graphicx}
\usepackage{hyperref}
\usepackage{cite}
\usepackage{minted}
\usepackage{wrapfig}
\usepackage{hyperref}
\renewcommand{\baselinestretch}{1.5}
\renewcommand{\figurename}{Abbildung}
\renewcommand{\contentsname}{Inhalt}

\newcommand{\picturesource}[2]{\item {\figurename} \ref{#1}: \href{#2}{#2}}

\author{Clara Maier, Nico Fröhlich}
\title{Seminarkurs - Alan Turing}

\begin{document}
\maketitle
\newpage
\emph{Hiermit versichern wir, die vorliegende Arbeit selbstständig und nur unter Zuhilfenahme der angegeben Quellen und Hilfsmittel angefertigt zu haben. Aus den angegebenen Materialien entnommene Inhalte und Zitate sind als solche kenntlich gemacht. Wir erklären, dass wir zu diesem Thema nicht schon einmal eine solche Arbeit angefertigt habe.}
\newpage
\section{Vorwort}
Der Vater des Computer; die Person, die den zweiten Weltkrieg wahrscheinlich um zwei Jahre verkürzte. Was gibt es da noch zu sagen? Alan Turing ist einer der bekanntesten und einflussreichsten Mathematiker des letzten Jahrhunderts. Er beeinflusst nicht nur die Mathematik und heutige Informationstechnik durch die Turingmaschine, sondern auch die Philosophie, durch zum Beispiel den Turing Test. In seiner meist zitierten Arbeit ging es sogar um Biologie. Am bekanntesten, nicht zuletzt durch den Film 'The Imitation Hame', ist jedoch seine Beteiligung am zweiten Weltkrieg, durch die Entschlüsselung der Enigma der Nazis. Seine Arbeit im Bereich Computertechnik, sowie in Philosophie, sein Papier über Maschinen und Intelligenz, sind mehr als nur interessant. Turing hat mit diesen Werken unsere heutiges Leben maßgeblich beeinflusst. In Kapitel \ref{historie} wird die Biografie rund um Alan Turing beleuchtet. Hierzu wird auch etwas auf den geschichtlichen Kontext eingegangen. Des Weiteren werden ein paar seiner Werke hier zeitlich eingeordnet. Darauf in Kapitel \ref{enigma} erfährt man alles rund um die Verschlüsselungsmaschine Enigma, wie sie funktioniert, aus welchen Teilen diese besteht und warum sie geknackt wurde. Des Weiteren wird auf die Geschichte der Enigma, so wie Turings Arbeiten an der Entschlüsselung mit Hilfe der Bombe eingegangen. In Kapitel \ref{turingmaschine} wird auf das theoretische Datenverarbeitungssystem von Alan Turing eingegangen. Abschließend in Kapitel \ref{turingtest} geht es um die Frage, wie man Menschen von Maschinen unterscheidet und ob Maschinen denken können. Wie wichtig diese Fragen geworden sind und noch weiter werden, sowie die Einflüsse Turings Arbeiten in der Philosophie, werden hier auch beleuchtet.
\tableofcontents
\newpage
\section{Historie}
Alan Mathison Turing wurde 1912, am 23 Juni in London geboren.Er starb am 7 Juni 1954 im Alter von 42 Jahren an seinem Wohnort in Wilmslow in Cheshire. Er machte seinen Abschluss an der Sherborne School in Dorset und besuchte anschließend ab Oktober 1931 das King's College in Cambridge, an dem er Mathematik studierte. Er beendete sein Studium 1934 und im März 1935 arbeitete er im Alter von 22 Jahren bereits als Dozent am King's College. 1936 veröffentlichte er eine seiner wichtigsten Werke "On Computable Numbers, with an Application to the Entscheidungsproblem", welche das Prinzip der Turing Maschine beinhaltet. Durch die Unterstützung von John von Neumann und Max Newman arbeiteten bereits mehrere Gruppen 1945 an der Umsetzung der Turing Maschine in Hardware, also an der Umsetzung des ersten Computers. Turing verließ 1936 Cambridge um seine Forschungen an der Princeton University weiter zu führen und veröffentlichte 1938 "Systems of Logic Based on Ordinals". 
Im Sommer 1938 kehrte Turing als Dozent an die King's zurück bis er beim Ausbruch des Krieges im September 1938 in den Bletchley Park zog, ein millitärisches Lager zur Entschlüsselung deutscher Nachrichten. Turings exzellente Arbeit hatte Kriegsentscheidente Konsequenzen. Turing war Hauptentwickler der Turing Bombe, eine extrem schnelle Entschlüsselungsmaschine. Im Nachhinein wird vermutet das die Arbeiten von Turing und seinen Kollegen den Krieg in Europa um mindestens 2 Jahre verkürzt hat.
Als der Krieg 1945 beendet war, ging Turing nach London ins National Physical Laboratory um seine Theorie der Turing Maschine in Hardware umzusetzen. Im Februar 1947 veröffentlichte er "Lecture on the Automatic Computing Engine" in der er als Erster etwas über Computerintelligenz schreibt und in seinem technischer Bericht 1948 "Intelligence Machinery" definiert er den Begriff 'Artificial Intelligence' kurz auch 'AI'. Im zwei Jahre später erschienenen Artikel "Computing Machinery and Intelligence" behandelt er den heute bekannten Turing Test, der ein Kriterium ist ob Maschinen denken können oder nicht.
\section{Enigma und Turing Bombe}
In diesem Abschnitt wird die Enigma und ihre Entschlüsselung thematisiert. Des weiteren wird auf die historischen Hintergründe eingegangen, auf ihre Funktionsweise sowie ihre Fehler.

\subsection{Enigma}
Die Enigma ist eine Verschlüsselungsmaschine, die im zweiten Weltkrieg von den Deutschen genutzt wurde um militärische Fernkommunikation zwischen ihren Truppen zu etablieren. Diese sollte ihnen mit unter einen entschiedenen Vorteil liefern. Dennoch hatte diese Verschlüsselung ein paar sehr große Probleme, die dazu führten, dass die Alliierten immer wieder sehr viele Informationen entschlüsseln konnten.\\
Hauptbestandteile der Enigma waren:
\begin{enumerate}
\item Eingabetastatur
\item Lichtanzeige
\item Steckbrett
\item Reflektor
\item drei oder vier Zahnräder
\end{enumerate}

\subsubsection{Eingabetastatur}
Eine zur Schreibmaschinen ähnliche Tastatur mit allen 26 Buchstaben des Alphabets, in die man die Nachricht eintippen konnte. Diese Tastatur war mit dem Steckbrett verknüpft.

\subsubsection{Lichtanzeige}
Eine Anzeige mit allen 26 Buchstaben des Alphabets, die einzeln beleuchtet werden und die verschlüsselte Nachricht anzeigen z.B ein H wird eingetippt und ein R wird beleuchtet.

\subsubsection{Steckbrett}
\label{sec:steck}
Das Steckbrett, welches hinten an der Enigma angebracht war, diente zur weiteren Verschlüsselung, da man hier bis zu 13 von 26 Buchstaben manuell miteinander verlinken konnte. Somit wurde z.B. ein eingegebenes A zu einem H und vice versa. Wenn kein Kabel eingesteckt wurde, dann wurde der Buchstabe einfach normal durch geschallten. Meist wurden nur 10 umgesteckt.

\subsubsection{Zahnräder}
\label{sec:rader}
Die Zahnräder dienten zum weiteren Verschlüsseln der Nachrichten und jedes Zahnrad hatte 26 Stufen für jeden Buchstaben des Alphabets. Wobei das Signal nun innerhalb des Rades durch neue Verkabelung auf einen anderen Buchstaben geleitet wurde (sog. rewiring). Das Signal wurde an das nächste Rad weiter geleitet, dort passiert das selbe nochmal. Es gab 8 verschiedene Arten von Rädern, diese besaßen pro Art immer die Selben Verkabelung innerhalb der Zahnräder. Das erste Zahnrad, welches die Eingabe passierte drehte sich bei jedem Tastendruck um einen Buchstaben weiter. Die nachfolgenden Zahnräder drehten sich immer dann um eines wenn sich das vorherige an seinem Überrollpunkt befindet. Dieser ist meist wenn sich das Rad einmal um sich selber gedreht hat, dass kann etwa bei M oder aber auch Z sein. Der Überrollpunkt konnte über eine Offset-Markierung an den Rädern eingestellt werden (sog. ring setting), so dass sich die Neuverkabelung um so viele Stellen wie eingestellt in eine Richtung weiter drehten. Diese wurden am Anfang jeder Nachricht einmal eingestellt. Am Anfang wurden diese Starteinstellungen über die Nachricht mitgeteilt, in dem man ein deutsches Wort mit 3 Buchstaben zweimal hintereinander gesendet hat. Was sich jedoch schnell als ziemlich schlechte Idee herausstellte und somit 1938 verboten wurde.

\subsubsection{Reflektor}
Der Reflektor schickte die Signale die von den Rollen kamen durch eine andere Leitung an den Rollen wieder zurück. Dieser konnte nie zum gleichen Buchstaben zurück senden sondern wurde wieder neu verkabelt.

\subsubsection{Beispiel}

\begin{figure}[H]
\centering
\includegraphics[scale=0.2]{Enigma_Maschine_Beispiel.jpg}
\caption{Beispiel Verkabelung}
\label{fig:enigma}
\end{figure}

Der eingegebene Buchstabe A geht erst einmal an das Steckbrett und ist hier mit dem Buchstaben M verbunden. Dieser wird dann an die Zahnräder weitergegeben, die den Buchstaben je nach Stellung verändern, hier von B zu J und schließlich F. Anschließend wird der Buchstabe wieder zurück durch die Zahnräder geleitet und kommt in diesem Beispiel als O heraus und wird wieder an das Steckbrett weitergeleitet. Der Buchstabe O ist hier mit H verbunden und so wird am Ende H in der Anzeige beleuchtet, und somit wurde A zu H verschlüsselt.

\subsection{Historie}
Turing beschäftigte sich bereits vor Beginn des zweiten Weltkriegs mit den Enigma Maschinen, allerdings waren diese von den Italienern und wesentlich simpler wie die Enigma Maschinen später genutzt von den Deutschen. Die erste Version der Enigma war frei verkäuflich und wurde eher privat genutzt da sie nicht viel Sicherheit bot. In dieser Version gab es lediglich 3 der in Sektion \ref{sec:rader} beschriebenen Räder, die in unterschiedlicher Reihenfolge eingesetzt werden konnten. Damit gab es lediglich $3! = 6$ Anordnungsmöglichkeiten. Zu dieser Zeit gab es außerdem kein, wie in Sektion \ref{sec:steck} aufgeführtes Steckbrett. Ein polnisches Verschlüsselungsteam unter der Leitung von Marian Rejewski legte den Grundstein und knackte die Verschlüsselungen der Enigma sehr schnell, dort wurden sie zum ersten mal mit Steckbrett Versionen konfrontiert. Diese knackten sie von Hand mit Hilfe von Lochpapieren. Deutschland wollte sich diese Verschlüsselungsmaschinen nun für militärische Zwecke zu nutzen machen, so entwickelten sie die nächste Version, die militärische Enigma. In dieser Version wurden zwei Zahnräder zur Auswahl hinzugefügt. Nun wurden also 3 aus 5 Zahnrädern genommen und in unterschiedlichen Anordnungen in der Enigma kombiniert. Diese wurden nachher großflächig für die deutsche Luftwaffe und Armee eingesetzt. Doch der Marine war das nicht genug. Unter Admiral Karl Dönitz wurde eine Enigma speziell für die Marine entwickelt, bei der zu erst 3 aus 8 Zahnrädern und später dann 4 aus 8 Zahnrädern für die Verschlüsselung verwendet wurden. Zu allem Überfluss drehten sich die Räder 6 bis 8 zweimal an ihrem Überrollpunkt. Damit explodierten die möglichen Kombinationen fast exponentiell. Bei Ausbruch des Krieges wurde Turing zusammen mit Neuntausend anderen Spezialisten in den Bletchley Park geholt. Dieser Park wurde zu einer wahren Entschlüsselungsfabrik, in dem später mit Hilfe der \emph{Bombe}, 39 Tausend Enigma Nachrichten im Monat entschlüsselt wurden. Die sogenannte \emph{Bombe} war eine automatisierte Maschine zur Entschlüsselung der Enigma Nachrichten. Diese wurde ursprünglich von den Polen bis zur Invasion entwickelt. Durch die Invasion jedoch übergaben sie die von den Polen \emph{Bomba} genannte Maschine an die Briten. Dort verbesserten sie die Maschine weiter und bauten im Zuge des Krieges zwischen 60 und 100 verschiedene \emph{Bombes} in Großbritannien und noch einmal mindestens so viel in der USA. Dort halfen sie den Briten beim dechiffrieren per Unterseekabel. In Bletchley Park wurde der Platz sehr eng, deswegen wurden um das eigentliche Gebäude mehrere Hütten gebaut, die sogenannten \emph{Huts}. Schnell bekamen die \emph{Huts} spezielle Aufgaben. So war Hut 8, der Arbeitsplatz von Turing, mit der Entschlüsselung der sehr schweren Marine Enigma betraut. Die normalen Enigma Texte der Armee und der Luftwaffe wiederum, wurden in Hut 6 unter der Leitung von Gordon Welchman, ein weiterer berühmter Statistiker, bearbeitet.  \cite{enigmaproblem1} \cite{theessentialturing}

\subsection{Schwachstellen}
Die Enigma hatte einige Schwachstellen, die im ersten Moment vielleicht nicht als solche erscheinen. Dazu gehörte einmal, dass der Reflektor die Signale nie auf den gleichen Buchstaben zurück leiten konnte. Dadurch konnte man versuchen bekannte Wörter wie zum Beispiel "Wetterbericht" unter den verschlüsselten Text zu halten. Wenn einer der Buchstaben in dem Wort mit dem verschlüsselten Text an der Stelle übereinstimmte, dann konnte es sich nicht um das Wort handeln. Ziel der Entschlüsselung war es heraus zu finden mit welchen der Räder und mit welchen Radeinstellungen sie die \emph{Bombe} bestücken mussten. Diese sollte dann automatisch die Nachrichten entschlüsseln. Des weiteren war bei den Rollen 6 bis 8 der Marine das zweite Überrollen ein Indikator dafür, dass ein solches Rad eingesetzt wurde. Dies konnte die Möglichkeiten sehr einschränken. Das oben genannte Problem mit dem doppelt senden eines deutschen Wortes mit 3 Buchstaben hat natürlich dafür gesorgt, dass hier über die Wahrscheinlichkeiten in der Sprache es sehr einfach war die Ringeinstellung zu erraten. Mit Hilfe des oben genannten Reflektor Problems minimierte es natürlich weiter die Möglichkeiten. Dieses Problem bemerkten die Deutschen und verbaten diese Übersendung von Ringeinstellungen. Des weiteren halfen gestohlene Codebücher und Insiderinformationen beim Entschlüsseln. Dazu kam, dass die Deutschen jeden Morgen  um Punkt 6 Uhr ihren Wetterbericht sendeten, denn dieser begann mit dem Wort \emph{Wetterbricht} und endete mit \emph{Heil Hitler}. \cite{enigmaflaw} \cite{enigmaproblem1}
\section{Turing Maschine}
\label{turingmaschine}
Die Turingmaschine wurde 1936 von Alan Turing entwickelt und ist rein theoretisch, denn eine Maschine exakt wie es Turing beschreibt wurde niemals gebaut, da dies auch nicht sinnvoll wäre. Turing entwickelte diese Theorie aufgrund des Entscheidungsproblem von David Hilber und Wilhelm Ackermann \cite{theessentialturing}. Die Turing Maschine ist die heutige Grundlage vieler Programmiersprachen, wie Java, C++ oder Python. Diese Programmiersprachen werden auch als \textit{"Turing complete"} bezeichnet, was so viel heißt wie, dass die Turing Maschine diese theoretischerweise emulieren kann.

\subsection{Deterministische Turingmaschine}
Die Theorie der deterministischen Turingmaschine beschreibt die Aktionen der Turingmaschine als etwas, dass von Anfang an nur einen einzigen klaren Weg hat, sodass man theoretischerweise vor Programmstart sagen kann, welchen Weg die Turingmaschine nimmt. Diese Theorie ist auf alle Programme anwendbar, die Turing complete sind.

\subsection{Nichtdeterministische Turingmaschine}
Die Theorie der nichtdeterministischen Turingmaschine beschreibt keinen klaren von Anfang an festgelegten Weg, sondern vielmehr ein Programm, dass viele verschiedene mögliche Wege hat um ans Ziel zu kommen. Dieses Modell ist allerdings rein theoretisch und nach dem heutigen Wissensstand nicht realisierbar.

\subsection{Funktion}
Die Turing Maschine beinhaltet lediglich zwei Bauteile.
\begin{enumerate}
	\item  Ein unendlich langes Band, dass in eine Kette von horizontalen Kästchen eingeteilt ist. Jedes dieser Kästchen kann entweder eine 0, eine 1 oder auch nichts beinhalten. Zudem kann das Band nach rechts und nach links verschoben werden.
	\item Ein Kasten, auch "Heap" oder auch "Scanner" genannt, welcher die Zahlen auf dem Band einlesen, löschen und ändern kann.
\end{enumerate}

\begin{figure}[hbtp]
	\centering
	\includegraphics[scale=1]{TuringmashinePicture.png}
	\caption{Beispielhafte Darstellung einer Turing Maschine\cite{theessentialturing}}
\end{figure}

\subsection{Beispiel}
Zur besseren Vorstellung, wird das Ganze an einem Beispiel erklärt. Nehmen wir mal an, wir wollen, dass die Turing Maschine für uns unendlich hoch zählt. So kann man dies mit nur zwei einfachen Befehlen ausführen.

\begin{itemize}
	\item Befehl 1: Bei einer 1, ändere diese in eine 0 und gehe nach links
	\item Befehl 2: Bei einer 0 oder einer Lücke, ändere diese in eine 1 und gehe zum letzten Bit der Zahl
\end{itemize}

Die Befehle für die Turing Maschine kann man sich also heute wie die Programme auf unseren Computern vorstellen.

\subsection{Codebeispiel}
Dieser Code aus Java macht praktisch genau dasselbe, wie das kleine Beispielprogramm der Turing Maschine. Das Einzige was auffällt ist, dass dieses Programm nur bis 1000 zählt, da unsere Turing Maschine einen unendlichen Speicher besitzt, anders als jeder PC auf der Welt.

\begin{minted}{java}
for (int i = 0; i < 1000; i++) {}
\end{minted}

In der for-Schleife wird zuerst die Variable i mit dem Datentyp Integer versehen und auf null gesetzt. Hinter dem ersten Semikolon steht dann die Bedingung, solange die Variable i kleiner wie 1000 ist. Hinter dem zweiten Semikolon steht dann die Anweisung dafür, was passiert solange die Bedingung erfüllt ist, das i++ bedeutet, dass die Variable i um eins hochgezählt wird. 

\section{Turingtest}
In diesem Abschnitt geht es um Turings Ausflug in die Philosophie und sein Werk "Computing Machinery and Intelligence". Dabei wird die Frage betrachtet ob Maschinen denken können und wie man eine Maschine von einen Menschen unterscheiden kann, sowie die Probleme die dabei entstehen und heutige Anwendungen.
\subsection{The Imitation Game}
Turing stellt eine bessere Frage in den Raum als die oben Angegeben. Um die Frage allerdings Stellen zu können muss ich das "Imitation Game" einführen. Das Spiel funktioniert wie Folgt. Lass A ein Mann sein, B eine Frau und C einen Detektiv. Der Detektiv (C) will also heraus finden wer welches Geschlecht hat. Ohne dabei die Sprache zu hören oder die Person zu sehen. Um dafür zu sorgen das dies Passiert gehen wir davon aus, dass der Detektiv in einem anderen Raum ist wie A und B, die einzige Kommunikation die besteht ist also Textbasierte. Also Versucht der C Fragen zu verwenden um mit Hilfe der Antworten der beiden Personen zu bestimmen welche weiblich und welche Männlich ist. Dies wendet Turing nun auf Maschinen an, dabei beschränkt er sich auf digital Computer. Es wird versucht auf äußerliche Fragen zu verzichten, da dies zu keinem eindeutigen Ergebnis führen würde. Die Maschine könnte ganz einfach ein eigenes Aussehen annehmen, und die Person kennt ihres, daher ist es schwierig anhand der Antworten eindeutig zu Unterscheiden. Deshalb wird versucht auf Logische und Fragen einzugehen. Beispiel: Subtrahiere 56 von 18934.\cite{computing} Diese Spiel wird nun nicht nur dazu genutzt eine Maschine von einem Menschen zu Unterscheiden sondern, es wird impliziert die Frage gestellt ob eine Maschine denken kann. Wenn man ihre Antworten nämlich nicht mehr voneinander unterscheiden kann muss man sich Gedanken darüber machen ob wir dann nicht Intelligentes Leben geschaffen haben könnten und wie wir diese dann Behandeln.
\subsection{Chinese room problem}
Eine der berühmtesten Problem des Turing Tests ist das chinese room problem. Dieses Problem stellt in Frage ob man mit dem Turingtest wircklich die Intelligenz eines Lebewesens feststellen kann. Dabei wird eine weitere Situation angenommen. Nehmen wir also an zwei Personen sitzen sich gegen über in einem Raum. Die eine Person ist eine Chinese die andere Person ein Deutscher. Auf dem Tisch liegt nun ein Deutsch-Chinesisch Wörterbuch. Die beiden Personen versuchen Miteinander zu Kommunizieren. Dazu nutzt der Deutsche ein Wörterbuch. Nun stellt sich die Frage, wenn der Deutsche nun seinen Satz auf Chineesisch übersetzt, ob er wirklich weiß was er sagt. Übertragen auf eine Maschine stellt sich nun ebenfalls die Frage, kann die Maschine wirklich wissen was sie tut und diese Aktionen auch selber hinterfragen. Darüber hinaus stellt sich die Frage wie Intelligent Maschinen wirklich werden können.
\subsection{Anwendungen in der Moderne}
Hier folgt eine kurze Beleuchtung des heutigen Einsatzfeldes des Turing Tests. Dieser wird in der Internetsicherheit sehr oft verwendet um automatisierte Anfragen zu filtern. Als Beispiel wird hier exemplarisch ReCaptcha von Google Beleuchtet. Es gibt aber genug andere sogenannte Captcha Methoden. Diese geben eine Art von Fragen an hierbei ist die Kommunikation allerdings nicht nur, wie im Turing, auf Textbasierte Kommunikation beschränkt. Im Gegenteil hier werden sogar bewusst Bilder eingesetzt, damit der zu testende Computer bzw. die Person die den Computer bedient, eine eine Frage zu bzw mit diesen Bildern beantworten muss. Dabei sind die Antworten auf die Eigentliche Frage nicht einmal so wichtig wie es einem im ersten Moment erscheint. Die Ironie, und eine weiter Abweichung, ist dabei das Mann hier gegen ein Neuralesnetzwerk "spielt". Dieses muss anhand der Vorliegenden Informationen bestimmen ob die Anfrage von einem Mensch kam oder ob es sich dabei nur um eine automatisch generierte Anfrage handelt. Wie genau dieses Netzwerk genau entscheidet weiß wahrscheinlich nicht mal Google. Wichtig ist aber welche Daten dem Netzwerk dabei zur Verfügung gestellt werden um seine Entscheidung zu treffen. Für weitere Infos verweise ich hier auf einen Blog-Eintrag (https://www.security-insider.de/wie-funktioniert-ein-captcha-a-683209/). Zu den Auszuwertenden Daten gehören unter anderem die Mausbewegung sowie die gesendeten Daten des Browsers (z.B. User-Agent, IP, Cookies).
\subsection{Auswirkungen auf Kunst und Gesellschaft}
Diese Grundsätzlichen Gedanken und die Frage ob Maschinen überhaupt denken können hat weitreichende Auswirkung auf die Moderne Kunst, darunter vor allem Computerspiele. Sehr gute Beispiele sind das gleichnamigen Spiel "The Turing Test" oder auch "The Talos Principle" in denen der Turing Test sehr häufig vorkommt. Es ist wichtig hier darauf aufmerksam zu machen, dass das Imitation Game heute noch für sehr viel Kopfzerbrechen unter den Wissenschaftlern, Philosophen so wie in der gesamten Gesellschaft auslöst. Wir werden bald an den Zeitpunkt kommen wo es wichtig sein wird, sich mit dieser Frage auseinander zu setzen. Weiterführend dazu siehe The future of the mind: Exploring machine consciousness von Dr. Susan Schneider\cite{explorecons}


\section{P versus NP}
\label{P versus NP}
P versus NP ist ein ungelöstes Problem der Mathematik wie auch der theoretischen Informatik, es behandelt die Frage, ob Probleme die man einfach überprüfen kann, auch einfach zu lösen sind oder auch ob P das Gleiche ist wie NP. Hierbei steht das P für polynomiell und NP für nicht-deterministisch polynomiell. Dieses Problem gehört zu den Millennium-Problemen und eine Lösung dieses Problems wird mit einer Millionen US-Dollar vom Clay Mathematics Institute in Cambridge belohnt.

\subsection{Beispiel}
Ein NP Problem für eine deterministische Turingmaschine ist das erraten eines mehrstelligen Passworts. Hierbei muss die Turingmaschine jede einzelne Kombination durchgehen und somit wird die benötigte Zeit um das Passwort zu erraten, mit der Länge des Passworts auch exponentiell länger. 
Nehmen wir mal an, dass das Passwort ohne Berücksichtigung von Groß- und Kleinbuchstaben nur die 26 Buchstaben des Alphabets zur Verfügung hat. Somit hätten wir bei einem Passwort von einem Buchstaben 26 Möglichkeiten, bei einer Länge von zwei Buchstaben bereits 676 Möglichkeiten und bei einer Länge von zehn Buchstaben mehr als 141 Billionen Möglichkeiten.
\\
Dagegen wird das Ganze zu einem P Problem, wenn das Passwort bekannt ist und nur noch die Richtigkeit überprüft werden muss. Diese Dauer der Überprüfung ist linear und somit für eine Turingmaschine in einer relativen kurzen Zeit zu erledigen.

\subsection{Was wäre wenn P = NP}
Würde jemand beweisen, dass P dasselbe ist wie NP, wäre die asymmetrische Kryptografie oder auch Public-Key-Kryptografie hinfällig, da diese auf dem Konzept beruht, das P nicht dasselbe ist wie NP. Somit wären die meisten elektronischen Sicherheitssysteme einfach zu knacken, da man sie dann in einer polynomischen Zeitspanne lösen könnte.
Andererseits würde es auch dafür sorgen, dass die Fahrwegberechnungen zwischen verschiedenen Punkten ebenfalls polynomiell werden und somit in kürzerer Zeit effizientere Fahrwege berechnet werden können.


\section{Lambda calculus}
Der Lambda calculus ist ein Datenverarbeitungskonzept erfunden von Alonzo Church, dem Professor von Alan Turing. Turing studierte unter Church einige Zeit an der Princeton Universität bevor er dann wieder zurück nach England ging. Im folgenden Abschnitt wird kurz auf Church eingegangen, danach wird Churches Werk, der Lambda calculus, dessen Gesetze, Notation und Beispiele behandelt. Hier wird ein Einblick in die Welt der funktionellen Programmiersprachen und ihrer Geschichte gegeben.
\subsection{Alonzo Church}
\subsection{Gesetze des Lambda calculus}
Der Lambda calculus besteht aus 3 Grundgesetzen.
\begin{itemize}
\item Funktionen können erstellt werden.
\item diese können/müssen Daten Ein und Ausgeben.
\item Man weiß nicht was die Funktionen tun.
\end{itemize}
Ausführlicher: Es muss ein Weg vorhanden sein eine Funktion zu definieren, dafür gibt es eine Mathematische Definition. Die Erzeugung einer Funktion ist aber nicht auf die Mathematische Notation beschränkt, ganz im Gegenteil, die Notationen im Modernen Lambda Programming weichen, nicht zuletzt aus technischen Gründen, stark von der Mathematischen Notation ab. Dabei müssen im originalen Lambda Ein- und Ausgaben definiert werden. Dies weicht im modernen Lambda ebenfalls von der originalen Definition ab. Hier können Ein- und Ausgaben definiert werden, müssen es aber nicht. Dies führt zu mehr Flexibilität, so können die Lambda Funktionen auch an stellen Eingesetzt werden wo nicht direkt Daten verarbeitet werden. Als letztes kennt man beim Ausführen der Funktion den Inhalt der Funktion nicht. Das ist der große unterschied zu anderen Mathematischen Funktionen.\cite{lambdacalculus}
\subsection{Lambda in Programmiersprachen}
Schauen wir uns zur Erläuterung ein Beispiel aus Java an.
Es gibt eine Funktionsdeklaration die der Implementation vorgibt welche Ein und Ausgänge
eine gewisse Lambda Funktion hat sowie ihren Namen.
\begin{minted}{java}
@FunctionalInterface
public interface TestFunctionalInterface {
	public int test(double d);
}
\end{minted}
Die Funktion wird nun als Interface übergeben und von einer ausführenden Funktion genutzt.
Was hier auffällt ist das die ausführende Funktion nicht weiß was die Funktion eigentlich tut.
\begin{minted}{java}
public void runTest(TestFunctionalInterface interf) {
	int testres = interf.test(89.1);
	// Mache etwas mit dem resultat
}
\end{minted}
Normalerweise müsste man dieses Interface überschreiben und dort dann implementieren was die Funktion tun würde. Hier gibt es allerdings eine separate schreib weise um eine Funktion zu definieren. Die wie folgt durch die Zeichen Kombination -> eingeleitet wird.
Rechts von dem sog. Lambda operator stehen die Inputs rechts die Operationen bzw. Outputs 
\begin{minted}{java}
public void main() {
	runTest(input -> (int)Math.ceil(input));
}
\end{minted}

\subsection{Mathematische Notation}
Für den Ursprünglichen Lambda calculus definiert von Alonzo Church gibt es auch eine Mathematische Notation die wie Folgt durch das Zeichen Lambda eingeleitet wird.
\begin{equation}
\lambda(x) = x * 2
\end{equation}
Die Ursprüngliche Notation ist weit aus Mathematischer als die in derzeitigen Programmiersprachen. Dennoch lassen sich damit beliebig Daten verarbeiten. .\cite{lambdacalculus}
\begin{equation}
\lambda
\end{equation}
So lässt sich z.B. ein boolean wert (also true/false oder 1/0) verarbeiten 
\footnote{Kapitel 4 von Nico}
\section{Church-Turing Thesis}
In diesem Kapitel geht es um die Church-Turing Thesis, was sie besagt und was für Auswirkungen sie hat. Außerdem wird im Zuge dessen die Turing Maschine und der Lambda calculus gegenübergestellt.
\subsection{Die These}
Die These besagt, dass alle möglichen Rechnungen bzw. Datenverarbeitungen mit dem Modell der Turing Maschine dargestellt werden können. Des weiteren ist der Lambda calculus mit der Turing Maschine gleich zu stellen, das heißt, dass ebenfalls jeder Datensatz mit Hilfe diese Paradigma verarbeitet werden können. Beide Systeme sind also nur Unterschiedliche Darstellungsweisen für Rechnungen bzw. Datenverarbeitung.
\cite{sep-church-turing}
\subsection{Gegenüberstellung}
Wie bereits erwähnt hat der Lambda calculus keinen internen "State", dies steht im Kontrast zur Turing Maschine diese hat einen internen "Speicher". Churchs Theorie basierte eher auf der Verarbeitung von ganz Zahlen bei der Speicherbetrachtung nicht im Vordergrund war. Damit war er allerdings ein paar Jahre zu früh, denn die Turing Maschine entwickelte sich schnell zum Standardmodell. Dies ist dem primär dem geschuldet, dass die Betrachtung von Speichern zu dieser Zeit und bis heute ein wichtiger Bestandteil der Informatik bzw. der Logischen Mathematik sind. Tatsächlich erkannte Gödel den Lambda calculus erst, an als er das Papier zur Turing Maschine sah und die Äquivalenz beider Systeme erkannte.\cite{sep-church-turing} In der Heutigen Zeit wiederum wird durch zunehmende Abstraktion und Komplexität der Lambda calculus wieder wichtiger. Er taucht vermehrt in den gängigen Programmiersprachen auf. So bekamen Java, C\# und C++ in der letzten Zeit vermehrt Lambda unterstützen. Dies ist besonders dadurch aufkommend, dass man sich keine Gedanken über den Heap bzw. die Speicher an sich machen muss. Dies kann die Arbeit und Komplexität sowie Fehleranfälligkeit reduzieren.
\subsection{Beispiele}
Widmen wir uns also dem oben bereits gezeigten Beispiel des Hochzählens. Mit Funktionen würden wir dies in JavaScript wie folgt schreiben.
\begin{minted}{javascript}
incrementor = (x) => incrementor(x + 1)
incrementor(0)
\end{minted}
Wie man hier schön sehen kann muss man sich nicht wie bei der Turing Maschine um Speicher kümmern was diese Darstellung unabhängiger vom eigentlichen Speichersystem macht. Wie dem auch sei, da Rekursion, also das selbst aufrufen einer Funktion komplizierter ist als es hier aussieht wird diese Funktion zwangsläufig irgendwann einen Fehler werfen. Bei der klassischen Methode ist es jedoch nicht so, hier ist der Hochzählvorgang lediglich durch die Eigentlichen Speicher begrenzt.
\begin{minted}{java}
int i = 0;
while(true) {
	i++;
}
\end{minted}
Wenn wir uns nun ein gegen Beispiel anschauen. In diesem Fall versucht man eine Dynamische liste zu sortieren. Hierzu die Klassische version. Es wird dabei ein Standard Bubble Sort Algorithmus eingesetzt.\newpage
\begin{minted}{java}
int temp;
for (int i = 1; i < arrayList.size(); i++) {
	for (int j = 0; j < arrayList.size() - i; j++) {
		if (arrayList.get(j) > arrayList.get(j + 1)) {
			temp = arrayList.get(j);
			arrayList.set(j, arrayList.get(j + 1))
			arrayList.set(j + 1, temp);    	
		}
	}
}
\end{minted}
Nun schauen wir uns dazu die Lambda Variante an.
\begin{minted}{java}
ArrayList<Integer> arrayList = new ArrayList<Integer>();
arrayList.sort((x, y) -> x - y);
\end{minted}
Wie man sieht ist dies viel simpler, da es hier keinerlei Darstellung von speichern und anderem gibt. Ich kann zwar den Algorithmus auch genauer darstellen mit lambda Funktionen, aber warum sollte man.

\section*{Bildquellen}
{\renewcommand\labelitemi{}
\begin{itemize}
\picturesource{fig:turing}{https://blog.sciencemuseum.org.uk/wp-content/uploads/2013/12/Alan-Turing-29-March-1951-picture-credit-NPL-Archive-Science-Museum1.jpg}
\picturesource{fig:enigmaset}{https://d1e4pidl3fu268.cloudfront.net/10fd7805-c5e4-
42ac-8f67-bc7d0a42e6d0/enigma\textunderscore machine.crop\textunderscore 720x540\textunderscore 68\%2C0.preview.jpg}
\picturesource{fig:rotorex}{https://de.wikipedia.org/wiki/Enigma-Walzen\# /media/Datei:Enigma\textunderscore rotor\textunderscore exploded\textunderscore view.png}
\picturesource{fig:enigma}{http://enigmamuseum.com/wp-content/uploads/2016/12/WiringDiagram-1024x776.jpg}
\item \figurename \ref{fig:recaptcha}: Selbst aufgenommen
\end{itemize}
}

\bibliography{main}
\bibliographystyle{ieeetr}

\end{document}
